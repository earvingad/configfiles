\documentclass[12pt,letterpaper,twoside ,openany]{book}
%\documentclass[10pt,letterpaper]{article}
\usepackage[urw-garamond]{mathdesign} %Tipo de letra Garamond
\usepackage[T1]{fontenc} %tipo de letra garamond
\usepackage[utf8]{inputenc}
\usepackage[activeacute,spanish,es-tabla]{babel}
%\usepackage{doi} %para que se pongan los enlaces de las referencias
%Configuración de referencias
%\DefineBibliographyStrings{spanish}{andothers={\&~al\adddot}}
%\renewbibmacro{y col.}{et al.}

\usepackage{csquotes}
\usepackage[backend=biber, autocite=superscript, citereset=chapter, refsection=chapter, style=numeric-comp, uniquename=false, sorting=none, natbib=true, url=false, isbn=false, doi=false, eprint=false]{biblatex}
\let\cite\autocite %to put cite as a superscript use autocite=superscript, plain will be [1-3]
\DeclareFieldFormat{labelnumberwidth}{\mkbibbold{#1}} %To match J. Mater. Chem. A bibligraphy report
\DeclareFieldFormat[article]{title}{} %to remove the journal title in bibliography
\DeclareFieldFormat[article]{journaltitle}{\mkbibemph{#1},}
\DeclareFieldFormat[article]{volume}{\textbf{#1}}
\DeclareFieldFormat{pages}{#1}
\DefineBibliographyStrings{spanish}{andothers={\textit{et~al}\adddot}} % Change "y col" for "et al."
\renewbibmacro{in:}{} % Remove "in:"

% Swap volume for year
\renewbibmacro*{volume+number+eid}{%
  \printfield{year}%
  \setunit{\addcomma\space}%
  \printfield{volume}%
  \setunit{\addcomma\space}%
}
% Remove year from bibliography
\renewbibmacro*{issue+date}{%
%\printfield{year}%
	\setunit{\addcomma\space}%
}
% Remove Month and number from bibliography
\AtEveryBibitem{%
      \clearfield{month}%
      \clearfield{number}%
}

%\addbibresource{library.bib}
\bibliography{/home/earving/Documentos/LaTex/library.bib}
%-----------------------------------
\decimalpoint %hace de las comas decimales puntos decimales.
%\usepackage{mathtools} %herramienta simbolos matemáticos
\usepackage[fleqn]{amsmath} %paquete paara ecuaciones matemáticas. fleqn, para alinear a la izquierda justificado.
\setlength{\mathindent}{0pt} %quita la sangría a las ecuaciones
\usepackage[pdftex]{graphicx} %paquete para insertar imágenes
\DeclareGraphicsExtensions{.pdf,.png,.jpg,.gif,} %formatos de imágenes soportados
\usepackage{multicol} %paquete para columnas
\usepackage{float} %paquete para columnas
\usepackage{authblk}
\usepackage{setspace} 
\renewcommand\Authands{ y }
\usepackage{titling} %paquete para reducir el espaciado del título
\setlength{\droptitle}{-3cm} %centimetros que se reducirá el espaciado del título
\usepackage{parskip}
\usepackage{multirow} %paquete para combinar columnas
\usepackage[font=small,labelfont=bf]{caption} %captions en pequeño y negrita.
\usepackage[left=2.54cm,right=2.54cm,top=1.5cm,bottom=1.27cm,letterpaper]{geometry} %Paquete de margen "includeheadfoot"
\usepackage{microtype}
\usepackage{fancyhdr}
\fancyhead[LE,RO]{\footnotesize \rightmark}
\fancyhead[LO,RE]{\footnotesize }
\fancyfoot[C]{}
\fancyfoot[RO, LE] {\thepage}
\pagestyle{fancy}
\renewcommand{\sectionmark}[1]{\uppercase{\markright{\thesection\ #1}}}
%\renewcommand{\chaptermark}[1]{ \markboth{}{} }
\fancypagestyle{bl}{
	\fancyhead{}
	\fancyhead[LE,RO]{\footnotesize CONGRESOS}
	\fancyhead[LO,RE]{\footnotesize CONGRESOS}
}
\fancypagestyle{art}{
	\fancyhead{}
	\fancyhead[LE,RO]{\footnotesize ARTÍCULOS}
	\fancyhead[LO,RE]{\footnotesize ARTÍCULOS}
}

\usepackage{lipsum}
%\usepackage{epigraph}
%\setlength{\beforeepigraphskip}{-1.3cm}
%\renewcommand{\epigraphsize}{\small}
%\renewcommand{\beforeepigraphskip}{1cm}
%\setlength{\epigraphwidth}{0.5\textwidth}
%\renewcommand{\epigraphrule}{0pt}
\usepackage{lettrine}
\addto\captionsspanish{\renewcommand\contentsname{Índice}}
\usepackage{hyperref} %make table of contents clickable
\usepackage{setspace} %Interlineado 
\onehalfspacing %interlineado 1.5
\usepackage{xcolor, soul} %color and highlight text
\sethlcolor{yellow} %color of highlighted text
\usepackage{pdfpages}
%----List of fogures and table by chapter
\usepackage{etoolbox}
\makeatletter
\newcommand{\thechaptername}{}
\newcounter{chapter@last@figure}
\newcounter{chapter@last@table}

\renewcommand{\chaptermark}[1]
            {
             \markboth{#1}{}
              \renewcommand{\thechaptername}{#1}
            }
\pretocmd{\caption}{
  \ifnum\pdfstrcmp{\@captype}{figure}=0
    \ifnum\value{chapter}=\value{chapter@last@figure}\else
      \addtocontents{lof}
        {\protect\numberline{\bfseries\thechapter\quad\thechaptername}}%
    \fi
  \fi
  \ifnum\pdfstrcmp{\@captype}{table}=0
    \ifnum\value{chapter}=\value{chapter@last@table}\else
      \addtocontents{lot}
        {\protect\numberline{\bfseries\thechapter\quad\thechaptername}}%
    \fi
  \fi  
  \expandafter\setcounter\expandafter{chapter@last@\@captype}{\value{chapter}}%
}{}{}
\makeatother
%------
%\newenvironment{abstract}%Pone resumen
%{\cleardoublepage\null\vfill \begin{center}
%\bfseries \abstractname \end{center}}%
%{\vfill\null}


\title{\textsc{Electrodeposición y caracterización de electrocatalizadores nanoestructurados para su aplicación en la reacción de evolución de oxigeno}}
\author{Earving Arciga Duran\thanks{earciga@cideteq.mx}}
\affil{CIDETEQ}

\begin{document}
\frontmatter
%\maketitle

\includepdf[pages={1}]{Imagenes/portada-tesis.pdf}
\pagestyle{plain}
\shipout\null
\includepdf[pages={1}]{Imagenes/Jurado.pdf}
\pagestyle{plain}
\shipout\null
\begin{flushright}
\null\vspace{\stretch{1}}
\textit{You know how people say\\``you can't live without love''?\\ Well, oxygen is even more important.}\\
$\sim$Dr. House.
\vspace{\stretch{2}}\null
\end{flushright}
\newpage
\pagestyle{plain}
\shipout\null
\include{resumen-espa}
\newpage
\include{resumen-ingles}
\pagestyle{plain}
\include{agradecimientos}
\shipout\null
\pagestyle{fancy}
%\let\cleardoublepage\clearpage
\tableofcontents
\listoffigures
\listoftables
\newpage
\pagestyle{plain}
\include{organizacion-tesis}
\mainmatter
\pagestyle{fancy}
\addcontentsline{toc}{part}{Generalidades}
\part*{Generalidades\\ \vspace*{-0.9cm}\noindent\rule{8cm}{0.4pt}\\ \vspace*{0.3cm} \raggedright \indent \normalsize \mdseries ¿Cuánto es $\frac{\text{la suma de los primeros} ~n~ \text{números primos}}{\text{la suma de los siguientes} ~n~ \text{números primos}}$?\hspace*{1cm} Considere que $\sum n_{\text{números primos}} = n^{2}$}
\begingroup
  \pagestyle{empty}
  \cleardoublepage
\endgroup
\include{introduccion}
\newpage
\include{Hipotesis-Objetivos}
\addcontentsline{toc}{section}{Referencias}
\printbibliography[title = {Referencias}]
\newpage
\include{metodosycondiciones}
\addcontentsline{toc}{section}{Referencias}
\printbibliography[title = {Referencias}]
\newpage
\shipout\null
\addcontentsline{toc}{part}{Electrocatalizador de Co$_{3}$O$_{4}$}
\part*{Electrocatalizador de Co$_{3}$O$_{4}$\\ \vspace*{-0.8cm}\noindent\rule{12cm}{0.4pt}\\ \vspace*{0.3cm} \raggedright \indent \normalsize \mdseries De la misma forma que puede demostrarse que $x=\frac{-b\pm\sqrt{b^{2}-4ac}}{2a}$ es una solución válida para la ecuación $ax^{2}\pm bx+c$, también es posible deducir que $\varphi=\frac{1\pm\sqrt{5}}{2}$ es una solución para la relación $\frac{a+b}{a}=\frac{a}{b}$.}
\shipout\null
\include{Co3O4}
\addcontentsline{toc}{section}{Referencias}
\printbibliography[title = {Referencias}]
\newpage
\addcontentsline{toc}{part}{Electrocatalizador de NiO}
\part*{Electrocatalizador de NiO\\ \vspace*{-0.8cm}\noindent\rule{12cm}{0.4pt}\\ \vspace*{0.3cm} \raggedright \indent \normalsize \mdseries Sea $x$ la respuesta a este problema, donde $x$ es un número entero real y $y$ la suma de sus dos dígitos. Calcule $2x-2y$, asumiendo que $x$ es la suma de su segundo dígito más diez veces su primer dígito.}
\begingroup
  \pagestyle{empty}
  \cleardoublepage
\endgroup
\include{NiO}
\addcontentsline{toc}{section}{Referencias}
\printbibliography[title = {Referencias}]
\newpage
\shipout\null
\addcontentsline{toc}{part}{Electrocatalizador de NiCo$_{2}$O$_{4}$}
\part*{Electrocatalizador de NiCo$_{2}$O$_{4}$\\ \vspace*{-0.8cm}\noindent\rule{12cm}{0.4pt}\\ \vspace*{0.3cm} \raggedright \indent \normalsize \mdseries Encuentre tres números enteros positivos tales que la suma de cualquiera de dos de ellos es un número al cuadrado (diferente en cada suma).}
\shipout\null
\include{NiCo2O4}
\addcontentsline{toc}{section}{Referencias}
\printbibliography[title = {Referencias}]
\newpage
\pagestyle{plain}
\include{Conclusiones-generales}
\clearpage
\include{perspectivas}
\newpage
\shipout\null
\part*{Participación en congresos\\ \vspace*{-0.8cm}\noindent\rule{12cm}{0.4pt}\\ \vspace*{0.3cm} \raggedright \indent \normalsize \mdseries \centering ¿Cuál es el principal valor de $i^{i}$ ? Preste atención a la fórmula de Euler $e^{ix}=\cos x + i\sin x. $}
\shipout\null
\pagestyle{bl}
\hspace*{0\textwidth}
\includegraphics[width=1\textwidth,height=1.5\textheight,keepaspectratio]{Imagenes/certificado-ISE.pdf}

\includegraphics[width=1\textwidth,height=1.5\textheight,keepaspectratio]{Imagenes/certificado-SMEQ.pdf}

\part*{Artículos publicados\\ \vspace*{-0.8cm}\noindent\rule{12cm}{0.4pt}\\ \vspace*{0.3cm} \raggedright \indent}
\shipout\null
\pagestyle{art}
\hspace*{0\textwidth}
\includegraphics[width=1\textwidth,height=1.5\textheight,keepaspectratio]{Imagenes/art2.pdf}

\includegraphics[width=1\textwidth,height=1.5\textheight,keepaspectratio]{Imagenes/art1.pdf}

\end{document}
